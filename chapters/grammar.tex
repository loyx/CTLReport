\chapter{CTL Grammar}

\section{Complete Grammar}
The grammar of CTL uses the EBNF(Extended Backus-Naur Form), where:
The notation of CTDL is a mixture of \href{https://en.wikipedia.org/wiki/Extended_Backus%E2%80%93Naur_form}{EBNF} and \href{https://en.wikipedia.org/wiki/Parsing_expression_grammar}{PEG}. 

\begin{grammar}
    \stm{ctlFile}{(\nter{topDef})\gstar}
    \stm{topDef}{\sttsp{CrowdTask} \nter{taskDef}}
        \orStm{\sttsp{class} \nter{classDef}}
    \stm{taskDef}{(\ntersp{constraintDef} \ORsp \ntersp{HpartDef} \ORsp \nter{funcDecl})\gstar}

    \emptyStm
    \stm{constraintDef}{\sttsp{Constraints} \sttsp{:} \tersp{NEWLINE} \tersp{INDENT} \nter{cDim}\gplus \gsp \ter{DEDENT}}
    \stm{cDim}{\ntersp{constraintDim} \sttsp{:} \nter{cSuite}}
    \stm{cSuite}{\gup{\ntersp{cStmt} \ORsp \tersp{NEWLINE} \tersp{INDENT} \nter{cStmt}\gplus \gsp \ter{DEDENT} } }
    \stm{constraintDim}{\sttsp{Temporal} \ORsp \sttsp{Spatial} \ORsp \sttsp{Precedence} \ORsp \sttsp{Date} }
        \orStm{\stt{Priority}}
    \stm{cStmt}{\ntersp{expression} (\stt{;})\gwild \gsp \ter{NEWLINE}}

    \stm{expression}{\ntersp{primary}}
        \orStm{\ntersp{expression} \bopsp{\stt{.} } \gup{\tersp{Identifier} \ORsp \nter{callOrCreate}}}
        \orStm{\ntersp{expression} \sttsp{[} \ntersp{expression} \stt{]}}
        \orStm{\ntersp{callOrCreate}}
        \orStm{\ntersp{expression} \bopsp{\gup{\sttsp{*} \ORsp \sttsp{/} \ORsp \stt{\%}}} \nter{expression}}
        \orStm{\ntersp{expression} \bopsp{\gup{\sttsp{+} \ORsp \stt{-}}} \nter{expression}}
        \orStm{\assocRightsp \tersp{Identifier} \bopsp{\stt{->}} \ter{Identifier}}
    \stm{primary}{\sttsp{(} \gup{\nter{listComp}}\gwild \gsp \sttsp{)}}
        \orStm{\sttsp{[} \gup{\nter{listComp}}\gwild \gsp \sttsp{]}}
        \orStm{\ter{Identifier}}
        \orStm{\nter{literal}}
    \stm{listComp}{\ntersp{expression} (\sttsp{,} \nter{expression})\gstar \gsp (\stt{,})\gwild}
    \stm{literal}{\nter{integerLiteral} }
        \orStm{\stt{INF}}
        \orStm{\nter{floatLiteral} }
        \orStm{\ter{TimeLiteral} }
        \orStm{\ter{DateLiteral}}
    \stm{integerLiteral}{\gup{\tersp{DecimalLiteral} \ORsp \tersp{HexLiteral} \ORsp \ter{OctLiteral}}}
    \stm{floatLiteral}{\gup{\tersp{FloatLiteral} \ORsp \ter{HexFloatLiteral}}}
    \stm{callOrCreate}{\tersp{Identifier} (\sttsp{(} \nter{expressionList}\gwild\gsp \stt{)})\gwild}
    \stm{expressionList}{\ntersp{expression} (\sttsp{,} \nter{expression})\gstar}

    \emptyStm
    \stm{HpartDef}{\sttsp{Hparticipant} \sttsp{:} \nter{hSuite}}
    \stm{hSuite}{\tersp{NEWLINE} \tersp{INDENT} \nter{abilityDecl}\gplus \ter{DEDENT}}
    \stm{abilityDecl}{(\stt{nullable})\gwild\gsp \nter{callOrCreate}}

    \emptyStm
    \stm{classDef}{\tersp{Identifier} \ntersp{classParamClauses} \nter{classBody}}
    \stm{classParamClauses}{(\sttsp{(} \ntersp{classParams} \stt{)})\gstar}
    \stm{classParams}{\ntersp{classParam} (\sttsp{,} \nter{classParam})\gstar}
    \stm{classParam}{(\sttsp{val} \ORsp \stt{var})\gwild\gsp \tersp{Identifier} \sttsp{:} \tersp{Identifier} (\sttsp{=} \nter{expression})\gwild}
    \stm{classBody}{\sttsp{\leftBrace} \nter{classBodyDecl}\gplus\gsp \stt{\rightBrace}}
    \stm{classBodyDecl}{(\ntersp{filedDecl} \ORsp \nter{methodDecl})}
    \stm{filedDecl}{\sttsp{var} \tersp{Identifier} \sttsp{:} \tersp{Identifier} (\sttsp{=} \nter{expression})\gwild}
    \stm{methodDecl}{\sttsp{def} \nter{procedure}}
    \stm{procedure}{\ntersp{signature} (\sttsp{:} \ter{Identifier})\gwild\gsp \sttsp{=} \nter{procBody}}
    \stm{procBody}{(\ntersp{expression} \ORsp \nter{block})}
    \stm{block}{\sttsp{\leftBrace} \nter{blockStm}\gstar\gsp \stt{\rightBrace}}
    \stm{signature}{\tersp{Identifier} (\sttsp{(} \nter{params}\gwild\gsp \stt{)})\gstar}
    \stm{params}{\ntersp{param} (\sttsp{,} \nter{param})\gstar}
    \stm{param}{\tersp{Identifier} (\sttsp{:} \ter{Identifier})\gwild\gsp (\stt{=} \nter{expression})\gwild}
    \stm{blockStm}{\ntersp{localVarDecl} \ter{NEWLINE}}
        \orStm{\ntersp{statement} \ter{NEWLINE}}
    \stm{localVarDecl}{\sttsp{var} \tersp{Identifier} (\sttsp{=} \nter{expression})\gwild}
    \stm{statement}{block}
        \orStm{\sttsp{if} \sttsp{(} \ntersp{expression} \sttsp{)}\ntersp{statement} (\sttsp{else} \nter{statement})\gwild}
        \orStm{\sttsp{break} \tersp{NEWLINE}}
        \orStm{\sttsp{continue} \tersp{NEWLINE}}
        \orStm{\ntersp{expression} \tersp{NEWLINE}}

    \emptyStm
    \stm{funcDecl}{\sttsp{compute} \nter{procedure}}
\end{grammar}

\section{Terminator Characteristics}
\subsection{Identifier}\label{Identifier}
An identifier consists of a letter followed by zero or more letters, digits, underscores, and single quotes.
\begin{grammar}
    \stm{Identifier}{\ntersp{Letter} \nter{LetterOrDigit}\gstar}
    \emptyStm
    \fragment{Letter}{\texttt{[a-zA-Z\_]}}
    \fragment{LetterOrDigit}{\nter{Letter}}
        \orStm{\texttt{[0-9]}}
\end{grammar}

\subsection{TimeLiteral}\label{TimeLiteral}
Time literals represent a point in time using the 24-hour system (\ref{timeformat}), with a minimum precision of 1 minute.
There should be no space between the hour and the colon, and the same between the minute and the colon
\begin{equation}\label{timeformat}
    hh:mm
\end{equation}
For example, the following is a legal time literal.

\begin{center}
    \begin{tabular}{c}
        00:12 \\
        20:30 \\
        23:59 \\
    \end{tabular}
\end{center}
And these things are not legal.

\begin{center}
    \begin{tabular}{c}
        00:62 \\
        25:30 \\
        24:00 \\
    \end{tabular}
\end{center}
\begin{grammar}
    \stm{TimeLiteral}{\lext{[0-2][0-9]':'[0-6][0-9]}}
\end{grammar}

\subsection{DateLiteral}\label{DateLiteral}
The date literal uses the form yyyy/mm/dd to represent a date, as shown in xx, with the following lexicon.
\begin{center}
    \begin{tabular}{c}
        2023/02/24 \\
        1999/03/30
    \end{tabular}
\end{center}

\begin{grammar}
    \stm{DateLiteral}{[0-9][0-9][0-9][0-9]'/'[0-1][0-9]'/'[0-3][0-9]}
\end{grammar}

\subsection{Coord} \label{Coord}
Coordinate uses WGS coordinates to represent a physical location, and wgs coordinates are classified as a \hyperref[FloatConst]{FloatConst}.
\begin{grammar}
    \stm{CoordLiteral}{\sttsp{(}  \stt{,} [+|-]\ter{FloatConst} \stt{)}}
    \stm{}{\gup{\sttsp{+} \ORsp \stt{-}}\gwild\gsp\ter{FloatConst}}
\end{grammar}
For instance, the \textbf{Coord}$(108.911, 34.153)$ represent the location of  xi'an, China in WGS.

\subsection{Numeric Literals}\label{FloatConst}\label{IntConst}
\subsubsection{Integer Literal}
\begin{grammar}
    \stm{DecimalLiteral}{(\sttsp{0} \ORsp \texttt{[1-9]} (\nter{Digits})\gwild\gsp \ORsp \stt{\_}\gplus\gsp \nter{Digits})} \label{DecimalLiteral}
    \fragment{Digits}{\lext{[0-9] ([0-9\_]* [0-9])?}}
    \stm{HexLiteral}{\sttsp{0} \lext{[xX] [0-9a-fA-F] ([0-9a-fA-F\_]* [0-9a-fA-F])?}}\label{HexLiteral}
    \stm{OctLiteral}{\sttsp{0} \stt{\_}\gstar\gsp \lext{[0-7] ([0-7\_]* [0-7])?}}\label{OctLiteral}
\end{grammar}
\subsubsection{Float Literal}
\begin{grammar}
    \stm{FloatLiteral}{(\ntersp{Digits} \sttsp{.} \ntersp{Digits} \ORsp \sttsp{.} \nter{Digits})\gsp \nter{ExponetPart}\gwild\gsp \lext{[fFdD]?}}
        \orStm{\ntersp{Digits} (\ntersp{ExponetPart} \lext{[fFdD]? | [fFdD]})} \label{FloatLiteral}
    \fragment{ExponetPart}{\lext{[eE] [+-]? } Digits}
    \stm{HexFloatLiteral}{\ntersp{Prefix} (\ntersp{HDigits} \stt{.}\gwild\gsp \ORsp \nter{HDigits}\gwild\gsp \sttsp{.} \nter{HDigits})\gsp \nter{HexEPart}} \label{HexFloatLiteral}
    \stm{Prefix}{\sttsp{0} \lext{[xX]}}
    \stm{HexEPart}{\lext{[pP] [+-]? } \ntersp{Digits} \lext{[fFdD]?}}
\end{grammar}


\subsection{Indent Related}\label{NEWLINE}\label{INDENT}\label{DEDENT}
In CTL, part of the syntax uses indentation to divide blocks of code. The details are handled in the grammar analysis.


